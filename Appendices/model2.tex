\documentclass[10pt,a4paper]{report}
\usepackage[utf8]{inputenc}
\usepackage[fleqn]{amsmath}
\usepackage{amsfonts}
\usepackage{amssymb}
\usepackage{graphicx}
\usepackage{multirow}
\usepackage{longtable}
\usepackage{array}
\usepackage{caption}
\begin{document}
\section*{Model}
\subsection*{Model parameters}
\begin{itemize}
\item $nbrTasks$
\item $nbrMachines$
\item $nbrTools$
\item $nbrTrays$
\item $nbrFixtures$
\item $nbrComponents$
\item $nbrOutputs$
\item $nbrConcurrentGroups$
\item $nbrOrderedGroups$
\item $tray_t$
\item $output_t$
\item $fixture_t$
\item $componentsUsed_t$
\item $mounting$
\item $taking$
\item $moving$
\item $putting$
\item $concurrentTasks_k$
\item $order_k$
\item $toolNeeded_t$
\item $changeToolDuration_{tool_1,tool_2}$
\item $duration_t$
\item $taskSubComponents_t$
\item $taskCompleteSubComponents_t$
\item $timeMatrix3D_{t1,t2}$
\end{itemize}


\subsection*{Variables}

\begin{equation}\label{eq:1}
nbrTasks \in \{1 , \ldots , 2^{32}-1\}
\end{equation}
The number of tasks to schedule

\begin{equation}\label{eq:2}
nbrMachines \in \{1 , \ldots , 2^{32}-1\}
\end{equation}
The number of machines available. A machine in our case is an arm. The YuMi robot has two arms.

\begin{equation}\label{eq:3}
nbrTools \in \{1 , \ldots , 2^{32}-1\}
\end{equation}
The number of tools available. These are the number of tools that can be fitted on an arm. In our example we use two tools, one to grip the button, top and ring components. And one to grip the bottom and switch component. These are identified by numbers ranging from 1 and up.

\begin{equation}\label{eq:4}
nbrTrays \in \{1 , \ldots , 2^{32}-1\}
\end{equation}
The number of trays available

\begin{equation}\label{eq:5}
nbrFixtures \in \{1 , \ldots , 2^{32}-1\}
\end{equation}
The number of fixtures available

\begin{equation}\label{eq:6}
nbrComponents \in \{1 , \ldots , 2^{32}-1\}
\end{equation}
The number of components used

\begin{equation}\label{eq:7}
nbrOutputs \in \{1 , \ldots , 2^{32}-1\}
\end{equation}
The number of outputs available

\begin{equation}\label{eq:8}
nbrConcurrentGroups \in \{1 , \ldots , 2^{32}-1\}
\end{equation}
The number of concurrent groups used. A concurrent group is a group of tasks that has to be performed at the same time.

\begin{equation}\label{eq:9}
nbrOrderedGroups \in \{1 , \ldots , 2^{32}-1\}
\end{equation}
The number of ordered groups used.

\begin{equation}\label{eq:10}
tasks = \{1 , \ldots , nbrTasks\}\end{equation}
The tasks to schedule.

\begin{equation}\label{eq:11}
machines = \{1 , \ldots , nbrMachines\}\end{equation}
The machines.

\begin{equation}\label{eq:12}
tools = \{1 , \ldots , nbrTools\}\end{equation}
The tools

\begin{equation}\label{eq:13}
components = \{1 , \ldots , nbrComponents\}\end{equation}
The components.

\begin{equation}\label{eq:14}
trays = \{1 , \ldots , nbrTrays\}\end{equation}
The trays.

\begin{equation}\label{eq:15}
fixtures = \{1 , \ldots , nbrFixtures\}\end{equation}
The fixtures.

\begin{equation}\label{eq:16}
outputs = \{1 , \ldots , nbrOutputs\}\end{equation}
The outputs.

\begin{equation}\label{eq:17}
concurrentGroups = \{1 , \ldots , nbrConcurrentGroups\}\end{equation}
The concurrent groups.

\begin{equation}\label{eq:18}
orderedGroup_k \subset tasks, \; k \in \{1 , \ldots , nbrOrderedGroups\}\end{equation}
Ordered group number $k$. These tasks come in the order in which they should be scheduled. The first one is the predecessor of the second one, the second one is the predecessor of the third one, and so on.

\begin{equation}\label{eq:19}
startTasks = \{nbrTasks+1 , \ldots , nbrTasks+nbrMachines\}\end{equation}
The start tasks. These work as source tasks/nodes for the first tasks in the schedule, i.e. they are the predecessors to the first tasks for each arm.

\begin{equation}\label{eq:20}
goalTasks = \{nbrTasks+nbrMachines+1 , \ldots , nbrTasks+nbrMachines*2\}\end{equation}
The goal tasks. Their predecessors are the last tasks for each arm.

\begin{equation}\label{eq:21}
allTasks = tasks \cup startTasks \cup goalTasks\end{equation}
All the tasks, including the start and goal tasks

\begin{equation}\label{eq:22}
tray_t \in trays \cup \{0\}, \; t \in tasks\end{equation}
The tray task $t$ uses. If none is used $tray_t = 0$

\begin{equation}\label{eq:23}
output_t \in outputs \cup \{0\}, \; t \in tasks\end{equation}
The output task $t$ uses. If none is used $output_t = 0$

\begin{equation}\label{eq:24}
fixture_t \in fixtures \cup \{0\}, \; t \in tasks\end{equation}
The fixture task $t$ uses. If none is used $fixture_t = 0$

\begin{equation}\label{eq:25}
componentsUsed_t \subset components, \; t \in tasks\end{equation}
The components used at task $t$

\begin{equation}\label{eq:26}
mounting \subset tasks\end{equation}
Set of tasks that mounts a component

\begin{equation}\label{eq:27}
taking \subset tasks\end{equation}
Set of tasks that takes a component

\begin{equation}\label{eq:28}
moving \subset tasks\end{equation}
Set of tasks that moves a component somewhere

\begin{equation}\label{eq:29}
putting \subset tasks\end{equation}
Set of tasks that puts a component somewhere

\begin{equation}\label{eq:30}
concurrentTasks_k \subset tasks, \; k \in concurrentGroups\end{equation}
Set of tasks needing concurrent execution

\begin{equation}\label{eq:31}
order_k \subset tasks, \; k \in orderedGroups
\end{equation} 
Array of tasks needed to be performed in a certain order on a single machine

\begin{equation}\label{eq:32}
order_{k,i} \in tasks, \; i \in \{1 , \ldots , |orderedGroup_k|\}, \; k \in orderedGroups
\end{equation}
The $i$:th task to be performed i order according to the $k$:th $orderedGroup$

\begin{equation}\label{eq:33}
toolNeeded_t \in tools, \; t \in tasks\end{equation} 
The tool needed for task $t$

\begin{equation}\label{eq:34}
changeToolDuration_{tool_1,tool_2} \in \{0 , \ldots , 2^{32}-1\}, \; tool_1,tool_2 \in tools
\end{equation} 
The duration of changing from $tool_1$ to $tool_2$

\begin{equation}\label{eq:35}
putting_c \subset putting, \; c \in components\end{equation}
Set of tasks that puts component $c$ somewhere

\begin{equation}\label{eq:36}
mounting_c \subset mounting, \; c \in components\end{equation}
Set of tasks that mounts component $c$

\begin{equation}\label{eq:37}
taking_c \subset taking, \; c \in components\end{equation}
Set of tasks that takes component $c$

\begin{equation}\label{eq:38}
moving_c \subset moving, \; c \in components\end{equation}
Set of tasks that moves component $c$ somewhere

\begin{equation}\label{eq:39}
orderedSet = \bigcup_{\forall k \in orderedGroups}order_k, \; orderedSet \subset tasks\end{equation}
The set of tasks member of ordered groups

\begin{equation}\label{eq:40}
usingMachine_t \in machines, \; t \in tasks\end{equation}
The machine task $t$ uses

\begin{equation}\label{eq:41}
pred_t \in allTasks, \; t \in allTasks\end{equation}
The predecessor task of task $t$

\begin{equation}\label{eq:42}
duration_t \in \{0 , \ldots , 2^{32}-1\}, \; t \in tasks\end{equation}
The duration of task $t$

\begin{equation}\label{eq:43}
timeMatrixDepth = \frac{n^2 - n + 2}{2}, \; n = nbrMachines\end{equation}
The depth of the 3D time matrix, i.e. the number of different transitions betweens different tools, independent of direction and a transition from tool to itself is the same transition for all tools


\begin{equation}
\begin{aligned}\label{eq:44}
timeMatrix3D_{t_{from},t_{to},k} \in &\{0 , \ldots , 2^{32}-1\}, \; t_{from} \in tasks \cup startTasks,  \\ 
&t_{to} \in tasks, \; k \in \{0 , \ldots , timeMatrixDepth\}
\end{aligned}
\end{equation}
The time to move from task $t_{from}$ to task $t_{to}$ changing tool according to $k$

\begin{equation}
\begin{aligned}\label{eq:45}
maxE = &(max(\{duration_t : t \in tasks\}) +  \\ 
&max(\{timeMatrix3D_{t_1,t_2,k} : \forall t_1 \in tasks \cup startTasks,  \\ 
&\forall t_2 \in tasks, \; \forall k \in \{0 , \ldots , timeMatrixDepth\}\}) \times nbrTasks
\end{aligned}
\end{equation}
Rough upper limit of the total schedule time. Assumes all move times take as long as the longest move time existing in the schedule. And likewise for the task durations.

\begin{equation}\label{eq:46}
start_t \in \{0 , \ldots , maxE\}, \; t \in allTasks\end{equation}
The start time for task $t$

\begin{equation}\label{eq:47}
end_t = start_t + duration_t, \; t \in allTasks\end{equation}
The end time for task $t$

\begin{equation}\label{eq:48}
makespan \in \{0 , \ldots , maxE\}\end{equation}
The makespan for the whole schedule, the time to minimize

\begin{equation}\label{eq:49}
moveDuration_t \in \{0 , \ldots , maxE\}, \; t \in allTasks\end{equation}
The duration of the move to task $t$ from its predecessor

\begin{equation}\label{eq:50}
moveStart_t \in \{0 , \ldots , maxE\}, \; t \in allTasks\end{equation}
The start time for the move to task $t$ from its predecessor

\begin{equation}\label{eq:51}
moveEnd_t = moveStart_t + moveDuration_t, \; t \in allTasks\end{equation}
The end time for the move to task $t$ from its predecessor

\begin{equation}\label{eq:52}
toolUsed_t \in tools, \; t \in allTasks\end{equation}
The tool used at task $t$

\begin{equation}\label{eq:53}
taskSubComponents_t \subset components, \; t \in tasks\end{equation}
The components that form the sub assemblies used in task $t$. The components can in them selves be sub assemblies

\begin{equation}\label{eq:54}
taskCompleteSubComponents_t \subset components, \; t \in tasks\end{equation}
The components that form the sub assemblies used in task $t$. This includes all sorts of components, being sub assemblies them selves or \emph{primitive} components.

\begin{equation}\label{eq:55}
subComponents_c \subset components, \; c \in components\end{equation}
The \emph{prmitive} components that form the component $c$. By being \emph{primitive}, the components in the set cannot be sub assemblies them selves.

\begin{equation}\label{eq:56}
taskOutOfRange_m \subset tasks, \; m \in machines\end{equation}
The tasks that cannot be reached by machine $m$


\subsection*{Filter}
\subsubsection*{Domain filter}


\begin{equation}
\begin{aligned}\label{eq:57}
maxMoveDurs_t = max(\{timeMatrix3D_{t,j,k} : \forall j \in tasks, \\
\forall k \in \{1 , \ldots , timeMatrixDepth\}, \; j \neq t\}), \; \forall t \in tasks
\end{aligned}
\end{equation}
The maximum duration for a move to task $t$


\begin{equation}
\begin{aligned}\label{eq:58}
minMoveDurs_t = min(\{timeMatrix3D_{t,j,k} : \forall j \in tasks, \\
 \forall k \in \{1 , \ldots , timeMatrixDepth\}, \; j \neq t\}), \; \forall t \in tasks
\end{aligned}
\end{equation}
The minimum duration for a move to task $t$

\begin{equation}\label{eq:59}
maxEnd = \sum_{\forall t \in tasks} duration_t + \sum_{\forall t \in tasks} maxMoveDurs_t\end{equation}
The upper limit of the schedule; all tasks is laid out after one after another and the duration between them is the maximum of the moves to them

\begin{equation}\label{eq:60}
minEnd = \left(\sum_{\forall t \in tasks} duration_t + \sum_{\forall t \in tasks} minMoveDurs_t\right) / nbrMachines
\end{equation}
The lower limit of the schedule; the total duration of each task is the duration of the task itself and the minimum duration of a move to the task, and the tasks are scheduled perfectly over all the machines

\begin{equation}\label{eq:61}
start_t \le maxEnd-duration_t \land end_t \le maxEnd, \; \forall t \in allTasks\end{equation}
Sets the upper limit for the start of each task to be the maximum end minus the duration for the task. Sets the end for each task to be the maximum end

\begin{equation}\label{eq:62}
end_t \ge duration_t + minMoveDurs_t, \; \forall t \in tasks\end{equation}
A task can start at its earliest at the time directly after the move to a task, therefore the end of a task can earliest happen after the duration of the task plus the shortest move to it

\begin{equation}\label{eq:63}
moveStart_t \le maxEnd-(duration_t+minMoveDurs_t), \; \forall t \in tasks\end{equation}
A move to a task can start at the latest $maxEnd$ but before the duration of the task and before at least the minimum of the move times to the task

\begin{equation}
\begin{aligned}\label{eq:64}
&moveDuration_t \le maxMoveDurs_t \; \land moveDuration_t \ge minMoveDurs_t, \\
&\forall t \in tasks
\end{aligned}
\end{equation}
The move duration for task $t$ is limited by $maxMoveDurs$ and $minMoveDurs$

\begin{equation}
\begin{aligned}\label{eq:65}
&moveEnd_t \le maxEnd - duration_t \; \land moveEnd_t \ge minMoveDurs_t, \\
&\forall t \in tasks
\end{aligned}
\end{equation}
The end of a move to a task can at the latest come at $maxEnd$ minus the duration of the task. The move to a task can at the earliest happen at time 0, so the end can earliest happen at the shortest move time to the task

\begin{equation}\label{eq:66}
makeSpan \le maxEnd \land makespan \ge minEnd\end{equation}
Limits the makespan

\begin{equation}
\begin{aligned}\label{eq:67}
&moveDuration_t \neq i, \\
&\forall t \in tasks,  \\
&\forall i \in \{0 , \ldots , maxMoveDurs_t\} / \{timeMatrix3D_{task,j,k} : \forall j \in tasks, \\
&\forall k \in \{1 , \ldots , timeMatrixDepth\}, t \neq j\}
\end{aligned}
\end{equation}
Limits the $moveDuration$ domains to only the values specified in the $timeMatrix3D$

\begin{equation}
\begin{aligned}\label{eq:69}
&moveStart_t \ge min(\{duration_{tt} + minMoveDurs_{tt} : \forall tt \in taking\}),\\
&\forall t \in tasks / taking\end{aligned}
\end{equation}
As the schedule has to start with a take task, the move to the other tasks can only start as early as after the shortest move to and execution of one of the take tasks

\begin{equation}
\begin{aligned}\label{eq:70}
&start_t \ge max(\{duration_{pt} + minMoveDurs_{pt} : \forall pt \in prevTasks\}), \\
&nbrMachines \ge |prevTasks|, \; 0 < |prevTasks|,\\
&prevTasks = \{task : \forall task \in tasks,\\
&componentCreated_{task} \in componentsUsed_t\},\\
&\forall t \in tasks
\end{aligned}
\end{equation}
$prevTasks$ are the tasks for which the task $t$ uses the component created at task $task$, hence the tasks in $prevTasks$ precedes task $t$. If the number of machines are greater than or equal to the number of task preceding task t, then the best scheduling is to do all tasks in parallel. If so the earliest task $t$ can start is greater or equal to the maximum of the preceding tasks

\begin{equation}
\begin{aligned}\label{eq:71}
&start_t \ge \left(\sum_{\forall pt \in prevTasks}duration_{pt} + minMoveDurs_{pt}\right) / nbrMachines, \\
&nbrMachines < |prevTasks|,  \\
&prevTasks = \{task : \forall task \in tasks, \; componentCreated_{task} \in componentsUsed_t\}, \\
&\forall t \in tasks
\end{aligned}
\end{equation}
$prevTasks$ are the tasks for which the task $t$ uses the component created at task $task$, hence the tasks in $prevTasks$ precedes task $t$. If the number of machines are less than the number of tasks preceding task t, then the best we can do is to divide the task times equally on all machines. If the tasks can be divided onto the machines so that the total length of the times on all machines are the same, that time will be equal to the sum$/nbrMachines$. If they don't match up the maximum of these times will be larger than the sum$/nbrMachines$.

\begin{equation}\label{eq:72}
\begin{aligned}
&end_t \le maxEnd - max(\{duration_{st} + minMoveDurs_{st} : \forall st \in succTasks\}), \\
&nbrMachines \ge |succTasks|, \; 0 < |succTasks|, \\
&succTasks = \{task : \forall task \in tasks,\\
&componentsUsed_t \subset taskCompleteSubComponent_{task}, \\
&componentsUsed_t \cup taskCompleteSubComponents_{task} \neq \emptyset\}, \\
&\forall t \in tasks
\end{aligned}
\end{equation}
$succTasks$ are the tasks that has the components used in task $t$ as subcomponents, hence the tasks in $succTasks$ succeeds task $t$. If the number of machines are greater than or equal to the number of task preceding task t, then the best scheduling is to do all tasks in parallel. If so the latest task $t$ can end is less than or equal to the maximum end of the schedule minus the longest of the succeeding tasks

\begin{equation}\label{eq:73}
\begin{aligned}
&end_t \le maxEnd - \left(\sum_{\forall st \in succTasks}duration_{st} + minMoveDurs_{st}\right) / nbrMachines, \\
&nbrMachines \le |succTasks|, \\
&succTasks = \{task : \forall task \in tasks, \\
&componentsUsed_t \subset taskCompleteSubComponent_{task}, \\
&componentsUsed_t \cup taskCompleteSubComponents_{task} \neq \emptyset\}, \\
&\forall t \in tasks
\end{aligned}
\end{equation}
$succTasks$ are the tasks that has the components used in task $t$ as subcomponents, hence the tasks in $succTasks$ succeeds task $t$. If the number of machines are less than the number of tasks preceding task t, then the best we can do is to divide the task times equally on all machines. If the tasks can be divided onto the machines so that the total length of the times on all machines are the same, that time will be equal to the sum$/nbrMachines$. If they don't match up the maximum of these times will be larger than the sum$/nbrMachines$.

\subsubsection*{Predecessor filter}
\begin{equation}\label{eq:74}
alldifferent(\{pred_t : \forall t \in tasks\})\end{equation}
Helps ensure that no two tasks can have the same predecessor

\begin{equation}\label{eq:75}
pred_{t1} \neq t2, \; t1 \neq t2, \; \forall t1, \forall t2 \in taking\end{equation}
No two taking tasks can be the predecessor of each other

\begin{equation}\label{eq:76}
pred_{t1} \neq t2, \; t1 \neq t2, \; \forall t1, \forall t2 \in putting\end{equation}
No two putting tasks can be the predecessor of each other

\begin{equation}\label{eq:77}
pred_{t1} \neq t2, \; t1 \neq t2, \; \forall t1, \forall t2 \in mounting\end{equation}
No two mounting tasks can be the predecessor of each other

\begin{equation}
\begin{aligned}\label{eq:78}
&pred_t \neq nonPred, \\
&\forall nonPred \in nonPredecessors, \\
&nonPredecessors = \{t2 : \forall t2 \in tasks, \\
&componentsUsed_t \subset taskCompleteSubComponents_{t2} \lor \\
&componentsUsed_t \subset subComponents_{componentCreated_{t2}}\} \\
&\forall t \in tasks
\end{aligned}
\end{equation}
A task $t$ cannot have task $t2$ as predecessor if task $t2$ uses a component, or creates a component, that the component task $t$ uses has as a subcomponent

\begin{equation}\label{eq:79}
pred_{putTask} \neq startTask, \; \forall putTask \in putting, \; \forall startTask \in startTasks\end{equation}
Since a component has to be taken before it can be put anywhere, put tasks cannot be first in the schedule

\begin{equation}\label{eq:80}
pred_{putTask} \neq startTask, \; \forall mountTask \in mounting, \; \forall startTask \in startTasks\end{equation}
Since a component has to be taken before it can be mounted anywhere, mount tasks cannot be first in the schedule

\begin{equation}\label{eq:81}
pred_{goalTask} \neq takeTask, \; \forall takeTask \in taking, \; \forall goalTask \in goalTasks\end{equation}
Since a schedule has to end with an assembly on the output, a take task cannot be at the end of the assembly

\begin{equation}
\begin{aligned}\label{eq:82}
&global\_cardinality(goalPreds, \; outputTasks, \; counts) \land \sum counts > 0, \\
&goalPreds = \{pred_{task} : \forall task \in goalTasks\}, \\
&outputTasks = \{task : \forall task \in tasks, \; output_{task} > 0\}, \\
&counts = \{i : \forall task \in outputTasks, \; i \in \{0 , \ldots , 1\}\}
\end{aligned}
\end{equation}
At least one of the output tasks has to be last on one of the circuits


\begin{equation}
\begin{aligned}\label{eq:83}
&global\_cardinality(takePreds, \; startTasks, \; counts) \land \sum counts > 0, \\
&takePreds = \{pred_{task} : \forall task \in taking, \; output_{task} = 0\}, \\
&counts = \{i : \forall task \in startTasks, \; i \in \{0 , \ldots , 1\}\}
\end{aligned}
\end{equation}
At least one of the take tasks, thats not on an output, has to be first on one of the circuits


\begin{equation}
\begin{aligned}\label{eq:84}
&pred_{putTask} \neq mountTask, \\
&\forall putTask \in putting_{comp}, \\
&\forall mountTask \in mounting_{comp}, \\
&\forall comp \in components
\end{aligned}
\end{equation}
If a set of tasks on a component involves a mount task and a put task, the predecessor of the put task cannot be the mount task


\begin{equation}
\begin{aligned}\label{eq:85}
&pred_{takeTask} \neq mountTask, \\
&\forall takeTask \in taking_{comp}, \\
&\forall mountTask \in mounting_{comp}, \\
&\forall comp \in components
\end{aligned}
\end{equation}
If a set of tasks on a component involves a mount task and a take task, the predecessor of the take task cannot be the mount task.


\begin{equation}
\begin{aligned}\label{eq:86}
&pred_{putTask} \le takeTask, \\
&\forall takeTask \in taking_{comp}, \\
&tray_{putTask} = tray_{takeTask}, \\
&\forall putTask \in putting{comp}, \\
&tray_{putTask} > 0, \\
&\forall comp \in components
\end{aligned}
\end{equation}
If a component has a put and take performed on it in a tray, the predecessor of the put task cannot be the take task.


\begin{equation}
\begin{aligned}\label{eq:87}
&pred_{putTask} \le takeTask, \\
&\forall takeTask \in taking_{comp}, \\
&fixture_{takeTask} = f \land componentsUsed_{putTask} \subset taskSubComponents_{takeTask}, \\
&\forall putTask \in putting{comp}, \\
&fixture_{putTask} = f, \\
&\forall f \in fixtures
\end{aligned}
\end{equation}
For every put action on a fixture, there is a take action. The predecessor of the put task cannot be the take task.

\begin{equation}
\begin{aligned}\label{eq:88}
&pred_{t1} \neq t2 \land pred_{t2} \neq t1, \\
&\forall t2 \in concurrentTasks_{group} / \{t1\}, \\
&\forall t1 \in concurrentTasks_{group}, \\
&\forall group \in \{1 , \ldots , nbrConcurrentGroups
\end{aligned}
\end{equation}
Concurrent tasks cannot be predecessor to each other.

\begin{equation}
\begin{aligned}\label{eq:89}
&pred_{t1} \neq t2, \\
&\forall t2 \in tasks, \; componentCreated_{t1} \in compinentUsed_{t2}, \\
&\forall t1 \in tasks, \; componentCreated_{t1} > 0
\end{aligned}
\end{equation}
Components cannot be used before they are created.

\begin{equation}
\begin{aligned}\label{eq:90}
&pred_{precTask} \neq t, \\
&\forall t \in tasks, \; precTask \neq t, \\
&componentsUsed_{precTask} \cup taskCompleteSubComponents_t \neq \emptyset, \\
&componentUsed_{precTask} \cup taskCompleteSubComponent_t \subset taskCompleteSubComponents_t, \\
&\forall precTask \in tasks
\end{aligned}
\end{equation}
Task using a component cannot execute before all the tasks having it as subcomponent.

\begin{equation}
\begin{aligned}\label{eq:91}
&pred_{postTask} \neq preTask, \\
&\forall predTask \in preTasks, \\
&\forall postTask \in postTasks, \\
&preTasks = \{preTask : \forall preTask \in tasks, \; componentsUsed_{preTask} \cap concSubComps \neq \emptyset\}, \\
&concSubComps = \bigcup_{\forall i \in concGroup}taskCompleteSubComponents_i, \\
&concComps = \bigcup_{\forall i \in concGroup}componentsUsed_i, \\
&\forall concGroup \in concurrentTasks, \; |concGroup| = nbrMachines
\end{aligned}
\end{equation}
If there is a set of concurrent tasks on a subset of tasks using as many machines as available, the tasks after the concurrent tasks cannot have the tasks before the concurrent tasks as predecessors.


\subsection*{Constraints}

\begin{equation}\label{eq:92}
end_t \le makespan, \; \forall t \in tasks\end{equation}
All ends has to be lesser than the total end

\begin{equation}\label{eq:93}
start_t = 0, \; \forall t \in startTasks \cup goalTasks\end{equation}
Start and goal tasks are not temporal tasks, i.e. they are timeless. Therefore, their start time is set to 0


\begin{equation}
\begin{aligned}\label{eq:94}
&usingMachine_{nbrTasks + m} = m \land usingMachine_{nbrTasks + nbrMachines + m} = m, \\
&\forall m \in machines
\end{aligned}
\end{equation}
The start tasks and goal tasks are assigned to machines, thereby there are start and goal tasks assigned to every machine. Because of the way start and goal tasks are created, the start tasks starts with number $nbrTasks + 1$, and the corresponding goal task for a start task can be accessed by $startTask + nbrMachines$.

\begin{equation}\label{eq:95}
usingMachine_t \neq m, \; \forall t \in tasksOutOfRange_m, \; \forall m \in machines\end{equation}
Setting the tasks that are out of range for each machine

\subsubsection*{Precedences}

\begin{equation}
\begin{aligned}\label{eq:96}
&end_{putTask} \le moveStart_{mountTask}, \\
&\forall putTask \in putting{comp}, \\
&\forall mountTask \in mounting_{comp}, \\
&\forall comp \in components
\end{aligned}
\end{equation}
If a set of tasks on a component involves a mount task and a put task, the put task has to come before the mount task


\begin{equation}
\begin{aligned}\label{eq:97}
&end_{takeTask} \le moveStart_{mountTask}, \\
&\forall takeTask \in taking_{comp}, \\
&\forall mountTask \in mounting_{comp}, \\
&\forall comp \in components
\end{aligned}
\end{equation}
If a set of tasks on a component involves a mount task and a take task, the take task has to come before the mount task


\begin{equation}
\begin{aligned}\label{eq:98}
&end_{putTask} \le moveStart_{takeTask}, \\
&\forall takeTask \in taking_{comp}, \\
&tray_{putTask} = tray_{takeTask}, \\
&\forall putTask \in putting{comp}, \\
&tray_{putTask} > 0, \\
&\forall comp \in components
\end{aligned}
\end{equation}
If a component has a put and take performed on it in a tray, the put has to come before the take.


\begin{equation}
\begin{aligned}\label{eq:99}
&end_{putTask} \le moveStart_{takeTask}, \\
&\forall takeTask \in taking_{comp}, \\
&fixture_{takeTask} = f \land componentsUsed_{putTask} \subset taskSubComponents_{takeTask}, \\
&\forall putTask \in putting{comp}, \\
&fixture_{putTask} = f, \\
&\forall f \in fixtures
\end{aligned}
\end{equation}
For every put action on a fixture, there is a take action. The put action has to come before the take action.

\begin{equation}
\begin{aligned}\label{eq:100}
&cumulative([moveStart_{task} : \forall task \in puts],\; [abs(end_{takes_i}-moveStart_{puts_i}) : \forall i \in \{1 , \ldots , |puts|\}], \\
&[1 : \forall i \in \{1 , \ldots , |puts|\}], 1), \\
&takes = [\operatorname*{arg\,min}_{\forall take \in takesForEachPut_p}(taskCompleteSubComponent_{take}) : \forall p \in \{1 , \ldots , |puts|\}], \\
&takesForEachPut = [\{take : \forall take \in taking, \; fixture_{take} = f, \\
&componentsUsed_{put} \subset taskCompleteSubComponent_{take}\} : \forall put \in puts], \\
&puts = [put : \forall put \in putting, \; fixture_{put} = f], \\
&\forall f \in fixtures
\end{aligned}
\end{equation}
The intervals between when components are put and then taken again cannot overlap on the same fixture.

\begin{equation}
\begin{aligned}\label{eq:101}
&start_{t1} = start_{t2} \land usingMahine_{t1} \neq usingMachine_{t2}, \\
&\forall t2 \in concurrentTasks_{group} / \{t1\}, \\
&\forall t1 \in concurrentTasks_{group}, \\
&\forall group \in \{1 , \ldots , nbrConcurrentGroups
\end{aligned}
\end{equation}
Concurrent tasks has to happen at the same time.

\begin{equation}
\begin{aligned}\label{eq:102}
&moveStart_{t2} \geq end_{t1}, \\
&\forall t2 \in tasks, \; componentCreated_{t1} \in compinentUsed_{t2}, \\
&\forall t1 \in tasks, \; componentCreated_{t1} > 0
\end{aligned}
\end{equation}
Components cannot be used before they are created.

\begin{equation}
\begin{aligned}\label{eq:103}
&end_{precTask} \leq moveStart_t, \\
&\forall t \in tasks, \; precTask \neq t, \\
&componentsUsed_{precTask} \cup taskCompleteSubComponents_t \neq \emptyset, \\
&componentUsed_{precTask} \cup taskCompleteSubComponent_t \subset taskCompleteSubComponents_t, \\
&\forall precTask \in tasks
\end{aligned}
\end{equation}
Task using a component cannot execute before all the tasks having it as subcomponent.

\begin{equation}
\begin{aligned}\label{eq:104}
&cumulative([start_{t} : \forall t \in fixtureTasks], \; [duration_t : \forall t \in fixtureTasks], \; [1 : t \in fixtureTasks], \; 1), \\
&fixtureTasks = [t : \forall t \in tasks, \; fixture_t = f], \\
&\forall f \in fixtures
\end{aligned}
\end{equation}
Tasks on the same fixture cannot overlap.

\begin{equation}
\begin{aligned}\label{eq:105}
&cumulative([start_{t} : \forall t \in trayTasks], \; [duration_t : \forall t \in trayTasks], \; [1 : t \in trayTasks], \; 1), \\
&trayTasks = [t : \forall t \in tasks, \; tray_t = tr], \\
&\forall tr \in trays
\end{aligned}
\end{equation}
Tasks on the same tray cannot overlap.

\begin{equation}
\begin{aligned}\label{eq:106}
&cumulative([start_{t} : \forall t \in outputTasks], \; [duration_t : \forall t \in outputTasks], \; [1 : t \in outputTasks], \; 1), \\
&outputTasks = [t : \forall t \in tasks, \; output_t = o], \\
&\forall o \in outputs
\end{aligned}
\end{equation}
Tasks on the same output cannot overlap.

\begin{equation}\label{eq:107}
Start_t \geq moveEnd_t, \forall t \in tasks\end{equation}
A task can only start after the move to it.

\subsubsection*{Predecessors}

\begin{equation}\label{eq:108}
moveStart_t \geq end_{pred_t}, \; \forall t \in tasks\end{equation}
A task has to start after its predecessor.

\begin{equation}\label{eq:109}
pred_{startTask} = startTask + nbrMachines - 1, \; \forall startTask \in startTasks / \{nbrTasks + 1\}\end{equation}
In order to create a circuit containing the sub circuits, for all start tasks, except the first one, the start tasks predecessor is the previous goal task.

\begin{equation}\label{eq:110}
pred_{nbrTasks + 1} = nbrTasks + nbrMachines \times 2\end{equation}
To complete the circuit, the first start tasks predecessor is the last goal task.

\begin{equation}\label{eq:111}
circuit(\{pred_t : \forall t \in tasks\})\end{equation}
The predecessors has to form a circuit.

\begin{equation}
\begin{aligned}\label{eq:112}
&pred_{mountTask} = takeTask, \\
&\forall takeTask \in taking_c, \; takeTask \notin orderedSet, \; puts = \emptyset, \\
&puts = \{p : \forall p \in putting_c, \; (fixture_p > 0 \land fixture_p = fixture_{mountTask}) \lor  \\
&(output_p > 0 \land output_p = output_{mountTask}) \lor  \\
&(tray_p > 0 \land tray_p = tray_{mountTask})\}, \\
&\forall mountTask \in mounting_c, \\
&\forall c \in components
\end{aligned}
\end{equation}
If a set of tasks on a component involves a mount and a take task, but no move tasks or put task on the same fixture, tray or output as the mount, the take task is the predecessor of the mount task.

\begin{equation}
\begin{aligned}\label{eq:113}
&pred_{putTask} = takeTask, \\
&\forall takeTask \in taking_c, \\
&\forall putTask \in putting_c, \; tray_{putTask} = 0, \\
&\forall c \in components, \; moving_c = \emptyset
\end{aligned}
\end{equation}
If a set of tasks on a component involves a put task not in a tray and a take task, and there is no moves involved, the take task has to be the predecessor of the put task.

\begin{equation}
\begin{aligned}\label{eq:114}
&pred_{orderedGroup_{k, i + 1}} = orderedGroup_{k, i}, \\
&\forall i \in \{1 , \ldots , |orderedGroup_k|-1\} \\
&\forall k \in \{1 , \ldots , nbrOrderedGroups\}
\end{aligned}
\end{equation}
Sets up the predecessors in accordance with the ordered groups.

\begin{equation}\label{eq:115}
usingMachine_t = usingMachine_{pred_t}, \; \forall t \in tasks \cup goalTasks\end{equation}
A task has to use the same machine as its predecessor.

\begin{equation}\label{eq:116}
moveDuration_t = timeMatrix3D_{pred_t, \; t, \; abs(toolUsed_t - toolUsed_{pred_t}) + 1}, \; \forall t \in tasks\end{equation}
Take tasks has to use the same tool as its predecessor or do a change first.

\begin{equation}\label{eq:117}
toolUsed_t = toolNeeded_t, \; \forall t \in tasks, \; toolNeeded \neq 0\end{equation}
Set the tool used for each task in accordance with $toolNeeded$.

\end{document}
