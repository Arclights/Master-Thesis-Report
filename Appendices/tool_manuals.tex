\chapter{File \& Tool Manuals}\label{app:tool_manuals}
The data and tools used in this thesis are available for free to download and use at \url{https://github.com/Arclights/Thesis-Tools}. In this section we will describe the tools and data. All the tools are written in Java and are run using the \texttt{java -jar} command.

\section{File Formats}
There are two files used to produce the data for the solvers; it is the XML file that describes the assembly, and the time matrix file that describes the time it takes an arm to move between two tasks. For complete versions of the files used in the thesis see the link above.
\subsection{Assembly XML}\label{sec:xml}
In order to more easily create the data needed by the solvers, we created an XML format that that is more easy to deal with, that later is translated into MiniZinc code. For an outline of the XML format see listing \ref{list:xml}. Note that although it contain the basic parts, it is not a legit assembly XML file as it is not complete.

All ids must be unique within the area they are used. So for example there can only be one tray called "tray1", but there could be a fixture called "tray1" as well, but that would be bad practise since it would be confusing. Ids used must also be declared before they are used again. For example a component must be declared through a \texttt{Component} tag before being referenced to in a task. The number of declarations can theoretically be infinite, but since everything will be represented by integers in the model we are practically limited to the limit of integers.

The \texttt{Output} tag, \texttt{Tray} tag and \texttt{Fixture} tag defines an output, a tray and a fixture respectively. The \texttt{Component} tag defines a component. All components used in the assembly needs to be defined, including the sub assemblies, since we treat sub assemblies as components in our model. We also describe what components make up a subcomponent using the \texttt{SubComponent} tag, it can both be regular components or sub assemblies.

To define tools we use the \texttt{tool} tag and to define machines we use the \texttt{Machine tag}

To define a task we use the \texttt{Task} tag. Together with the id we specify the length of the task in some time unit. Inside the tag we declare the properties of the task. If the task is performed in a tray, we specify the tray used. The same goes for fixture. Only one of them can be declared at a time, since we cannot be at a tray and a fixture at the same time. We specify which components are used in the task. There can be multiple components associated with a task, but it is limited by the number of machines available. Although the translation program does not check whether or not this limit is exceeded. If there is a component created at the task we specify if by the \texttt{ComponentCreated} tag. There can only be a most one component created per task and can only occur in tasks where the action is mounting, but this is not checked by the translator either. If there is a particular tool needed for the task we specify it with the \texttt{ToolNeeded} tag. We specify what kind of action a task is using the \texttt{Action} tag.

To define a set of tasks that comes in an ordered group we use the \texttt{OrderedGroup} tag. The order of them listed in the XML file is the order in which they will be scheduled. There can be multiple ordered groups specified.

To define a set of task that needs to be performed concurrently we use the \texttt{ConcurrentGroup} tag. The order in the XML does not matter.

To define a set of tasks out of range of a machine we use the \texttt{TasksOutOfRange} tag. The id is the id of the machine that the tasks are out of range for. This is not declaring the machine, so it needs to be defined previously.

To define the tool change durations we use the \texttt{ToolChangeDurations} tag and for each tool change we want to define we use the \texttt{Change} tag. And we supply the tool we are changing from, the tool we are changing to and the duration it takes to perform.

\begin{figure}
\lstinputlisting[label=list:xml,captionpos=b,caption=The basic parts of the assembly XML. This is not a legit assembly file.,tabsize=3,frame=L,basicstyle=\footnotesize\ttfamily]{Listings/assembly.xml}
\end{figure}

\subsection{Time Matrix}\label{sec:time_matrix}
To supply the time it takes to move between the tasks we provide a time matrix using a time matrix. A row represents the task we are going from and the column represents the task we are going to. There are as many columns as there are tasks in the XML file, but there are one more rows than there are columns. This is to account for the starting position of the machines.

The file format is a comma separated values file (CSV). To identify which row and column belongs to the correct task, each row and column starts with the corresponding id from the XML file. The row for the start task must be named \emph{Start}. The times from every task to the place where the tools are changed and vice versa needs to be included as a column and a row with the name \emph{Change tool}. The file needs to be separated with semicolons in order to allow for commas in the ids. This means that the ids used in the assembly cannot have semicolons in them. The values for the times can be decimal values, they will be rounded of after the calculations by the translator. Figure \ref{fig:time_matrix} shows an example of the structure of the time matrix.

\begin{figure}[h]
\begin{tabular}{c|c|c|c|c}
& Take top & Put top in fixture & \dots & Mount button on top\\\hline
Start & 3 & 7 & \dots & 7\\\hline
Take top & 0 & 8 & \dots & 8\\\hline
Put top in fixture & 8 & 0 & \dots & 0\\\hline
$\vdots$ & $\vdots$ & $\vdots$ & $\ddots$ & $\vdots$\\\hline
Mount button on top & 8 & 0 & \dots & 0
\end{tabular}
\caption{Example of time matrix structure}
\label{fig:time_matrix}
\end{figure}

\section{AssemblyConv}\label{sec:assemblyConv}
\texttt{AssemblyConv} is the program used to convert the XML file and the time matrix file into data for the solver. It takes the matrix file and the XML as arguments and produces a MiniZinc file. For a detailed description of the syntax run the program without parameters.

\section{SchedPrinter}
\texttt{SchedPrinter} is used for visualising the outputted assembly from the solvers. It creates a Gant diagram in ASCII and outputs it to the screen. To get it to a file one can simply pipe it to one and it can then be shown in a regular text editor. But make sure that text wrapping is turned of as it will interfere with the visualisation. It takes a text file containing the output from the solver. This can easily be obtained by piping the output from a solver into a file. The program also takes an argument whether the text file provided is in the format that the JaCoP solver provides or the format G12 provides. The solvers using the G12 format is G12/FD, Gecode and or-tools, and the solvers using the JaCoP format is JaCoP and Choco3. The output format for Opturion CPX is nknown since we have not been able to run the model on it. When printing from the JaCoP format, the user has to supply the \texttt{.dzn} file so the printer can know the name of the tasks. For a detailed description of the syntax run the program without parameters or with \texttt{-h} as parameter. The output of the tasks with duration 0 will look weird, but that is because we cannot fit any characters in a box with a width of 0.

\section{FZNstat}
To obtain the statistics used in chapter \ref{cha:eval} we use \texttt{FZNstat}. The program does not take any parameters but will go over all the \texttt{.fzn} files in the directory it is run in and produce a result file for each of the \texttt{.fzn} files. The names of the result files are the same as the \texttt{.fzn} files but with \texttt{\_stat} at the end. So for example \texttt{jacop.fzn} will get the result file \texttt{jacop\_stat}. The result files will contain a little bit more information about individual constraints than what was presented in chapter \ref{cha:eval}.