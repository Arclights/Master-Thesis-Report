\chapter{Approach}\label{cha:approach}

\section{Constraint Programming}
Constraint programing is a \emph{declarative} paradigm. This means that in contrast to \emph{imperative} paradigm languages, such as C or Java, the focus of solving problems using constraint programming is on specifying the problem and not the algorithm to solve it. However, \emph{declarative} languages, such as Java and C, can be used as a framework Constraint Programming, as in Jacop, or-tools, and others. One specifies the \emph{domain variables}, or simply variables, and \emph{constraints}. Variables have domains of values, meaning they can take any value in their domain. Variables can often be, depending on language and solver, either integers, floating-points, boolean or symbolic, symbolic being a text or label. For example a symbolic variable representing a week would have the domain\\
$\{Monday,Tuesday, Wednesday, Thursday, Friday, Saturday, Sunday\}$, while a integer one could have $\{0,1,2,3,4,5,6\}$

\subsection{Constraints}
Constraints are set up as relationships between the variables, and thereby limiting the domains of the variables. Integer domains are often used for variables, so for the rest of this section we will assume variables have integer domains. For this domain the following function symbols can be used: $+$, $\times$, $-$ and $\div$. The constraint relation symbols are $=$, $<$, $\leq$, $>$, $\geq$. Together with the function symbols and the constraint relation symbols, one can create simple constraint, called \emph{primitive constraint}. An example of a primitive constraint is $X < Y$, i.e.\ the values in $X$'s domain has to be lower than in $Y$'s. Primitive constraints can be used to create more complex constraints using the conjunctive connective $\land$. An example of this is $X < Y \land Y < 10$, i.e.\ $Y$ has to be less than $10$ and $X$ has to be less than $Y$. Since all constraints have to hold when the model is evaluated, all constraints are implicitly joined by a conjunction. The disjunctive connective $\lor$ is also available and can be used in the same way as $\land$. 

For example, lets assume we have a problem with two variables, $X$ and $Y$, $X = 4$ and $Y = \{1..10\}$. Here $X$ has the value $4$ and can thereby only take the value $4$. $Y$ on the other hand can take the values $1$ to $10$. This means a solution to this problem can be $X = 4$ and $Y = 1$ or likewise $X = 4$ and $Y = 5$ , they are equally correct.\\
On this problem we can impose a constraint, for example $Y > X$. Now we have set the constraint that $Y$ needs to be larger than $X$. And since $X$ has a fixed known value we can directly see that $Y > 4$, since $x = 4$. Now with this constraint, we can get rid of the lower part of $Y$'s domain and now $Y = \{5..10\}$ instead. And now a viable solution can be $X = 4$ and $Y = 7$, but not $X = 4$ and $Y = 3$.

\begin{figure}
  \documentclass{standalone}
\usepackage{tikz}
\usetikzlibrary{calc, shapes, backgrounds}
\usepackage{standalone}
\usepackage{amsmath, amssymb}
\pagecolor{olive!50!yellow!50!white}
\begin{document}
\tikzset{
  leaf/.style = {draw=none,label = center:\textsf{$\vdots$}},
  labels/.style={midway, sloped, above, yshift=-10pt}
}
\begin{tikzpicture}
[
    scale = 0.75, transform shape, thick,
    every node/.style = {draw, circle, minimum size = 5mm, line width = 1pt, align=center},
    grow = down,  % alignment of characters
    level 1/.style = {sibling distance=6cm},
    level 2/.style = {sibling distance=2cm}, 
    level 3/.style = {sibling distance=1cm}, 
    level distance = 3 cm
  ]
 \node (Start){}
   child {   node [] (A) {}
     child { node [] (D) {}
       child { node [leaf] (M) {}}
       child { node [leaf] (N) {}}
     }
     child { node [] (E) {}
       child { node [leaf] (O) {}}
       child { node [leaf] (P) {}}
     }
     child { node [] (F) {}
       child { node [leaf] (Q) {}}
       child { node [leaf] (R) {}}
     }
   }
   child {   node [] (B) {}
     child { node [] (G) {}
       child { node [leaf] (S) {}}
       child { node [leaf] (T) {}}
     }
     child { node [] (H) {}
       child { node [leaf] (U) {}}
       child { node [leaf] (V) {}}
     }
     child { node [] (I) {}
       child { node [leaf] (X) {}}
       child { node [leaf] (Y) {}}
     }
   }
   child{ node(C){}
     child { node [] (J) {}
       child { node [leaf] (Z) {}}
       child { node [leaf] (A1) {}}
     }
     child { node [] (K) {}
       child { node [leaf] (B1) {}}
       child { node [leaf] (C1) {}}
     }
     child { node [] (L) {}
       child { node [leaf] (D1) {}}
       child { node [leaf] (E1) {}}
     }
   };

  
  % Labels
  \begin{scope}[nodes = {draw = none}]
    \path (Start) -- (A) node [labels] {Pick $X$};
    \path (Start) -- (B) node [labels] {Pick $Y$};
    \path (Start) -- (C) node [labels] {Pick $Z$};
    \path (A)     -- (D) node [labels] {$X=1$};
    \path (A)     -- (E) node [labels] {$X=3$};
    \path (A)     -- (F) node [labels] {$X=6$};
    \path (B)     -- (G) node [labels] {$Y=5$};
    \path (B)     -- (H) node [labels] {$Y=9$};
    \path (B)     -- (I) node [labels] {$Y=8$};
    \path (C)     -- (J) node [labels] {$Z=6$};
    \path (C)     -- (K) node [labels] {$Z=3$};
    \path (C)     -- (L) node [labels] {$Z=5$};
    \path (D)     -- (M) node [labels] {Pick $Y$};
    \path (D)     -- (N) node [labels] {Pick $Z$};
    \path (E)     -- (O) node [labels] {Pick $Y$};
    \path (E)     -- (P) node [labels] {Pick $Z$};
    \path (F)     -- (Q) node [labels] {Pick $Y$};
    \path (F)     -- (R) node [labels] {Pick $Z$};
    \path (G)     -- (S) node [labels] {Pick $X$};
    \path (G)     -- (T) node [labels] {Pick $Z$};
    \path (H)     -- (U) node [labels] {Pick $X$};
    \path (H)     -- (V) node [labels] {Pick $Z$};
    \path (I)     -- (X) node [labels] {Pick $X$};
    \path (I)     -- (Y) node [labels] {Pick $Z$};
    \path (J)     -- (Z) node [labels] {Pick $X$};
    \path (J)     -- (A1) node [labels] {Pick $Y$};
    \path (K)     -- (B1) node [labels] {Pick $X$};
    \path (K)     -- (C1) node [labels] {Pick $Y$};
    \path (L)     -- (D1) node [labels] {Pick $X$};
    \path (L)     -- (E1) node [labels] {Pick $Y$};   
  \end{scope}
\end{tikzpicture}
\end{document} 
%     without .tex extension
  % or use \input{mytikz}
  \caption{The beginning of the search space for the variables $X$, $Y$, $Z$, where $X=\{1,3,6\}$ $Y=\{5,9,8\}$ $Z=\{6,3,5\}$}
  \label{fig:search_space}
\end{figure}

\subsection{Global constraints}
Global constraints are constraints that sets up a relation between an non-fixed number of variables and they can be reduce to a set of simpler binary constraints\\\cite{global_const}. They are also context independent, meaning they can be used in many contexts and does not take into account what context it is in\\\cite{global_constraint_catalogue}. This makes them quite convenient to use when modeling and we are using a couple of global constraints that are listed below.

The general discription of the global constraints in this section come from the \emph{Global Constraint Catalogue} \cite{global_constraint_catalogue} and the adaption in MiniZinc is from MiniZinc:s global constraints listing \cite{mz_global_constraints}.

\subsubsection{All Different Constraint (\texttt{allDifferent})}
The \texttt{allDifferent} constraint is pretty straightforward. It takes a collection \texttt{VARIABLES} as argument and enforces all variables in \texttt{VARIABLES} to assume distinctly different values, i.e. all values will be different.
\\
In MiniZinc, \texttt{VARIABLES} is called \texttt{x} and consists of a set of variables.

\subsubsection{Circuit Constraint (\texttt{circuit})}
The \texttt{circuit} constraint takes a collection of nodes \texttt{NODES}, where each node has an index and a successor. \texttt{circuit} enforces the nodes to form a \emph{Hamiltonian circuit}.

In MiniZinc, \texttt{NODES} is called \texttt{x} and consists of an array with variables where the index in the array is the index for the variable in that place and the successor value is the value of the variable.

\subsubsection{Cumulative Constraint (\texttt{cumulative})}
The \texttt{cumulative} constraint takes two arguments, a collection of tasks \texttt{TASKS} and a limit \texttt{LIMIT} for how many tasks can overlap simultaneously. The tasks have 4 attributes; \texttt{origin}, \texttt{duration}, \texttt{end} and \texttt{height}. The attributes are pretty self explanatory; \texttt{origin} is where in time the task starts, this is the value the constraint tries to determine, \texttt{duration} is the duration of the task, the \texttt{end} is at which time the task ends. \texttt{Height} might not be as straight forward though, it means how many resources the task consumes. \texttt{LIMIT} is the limit for how many resources there are available. So if we have a limit of $5$ resources and we have two tasks which consume $3$ and $2$ resources respectively, they can execute simultaneously. But not if the limit were $3$.

In MiniZinc, \texttt{TASKS} is split into $3$ arguments; \texttt{s}, \texttt{d} and \texttt{r}. Each of them is an array where the index represents a task. The \texttt{end} variable is skipped and is implied by the \texttt{s} and \texttt{d} arguments. The \texttt{r} variable in MiniZinc represents the \texttt{height} variable.

\subsubsection{Global Cardinality constraint (\texttt{global\_cardinality})}
The \texttt{global\_cardinality} constraint takes two arguments; a collection of variables \texttt{VARIABLES} and a collection of values \texttt{VALUES}. Each element in \texttt{VALUES} has two attributes; \texttt{val} and \texttt{noccurence}. The constraint ensures that \texttt{val} is covered \linebreak\texttt{noccurence} times by the elements in \texttt{VARIABLES}.

In MiniZinc, \texttt{VARIABLES} is split into two arguments; \texttt{cover} and \texttt{counts}, where \texttt{cover} corresponds to \texttt{val} and \texttt{counts} corresponds to \texttt{noccurence}. Both \texttt{val} and \texttt{counts} are arrays where the index identifies the element.

\subsection{Propagation}
To make sure 

\subsection{Solver}
A constraint programming program consists of many of these constraints and variables. When the solution is specified in a model, a \emph{solver} runs the model. The goal of the solver is to satisfy all the constraints, i.e. set the domains of the variables so that they all follow the relationships of the constraints. This is called the \emph{constraint satisfaction problem}, and can  be defined as a triple $\langle Z,D,C \rangle$. $Z=\{x_1 \ldots x_n\}$ is a finite set of all the variables in the solution, $D(x_i), \; x_i \in Z$ is a set representing the domain of values the variable $x_i$ can assume, $C$ is the set of constraints imposed on the variables in $Z$. In order to satisfy all constraints imposed, the solver performs a \emph{search} on the space of possibilities, i.e.\ the \emph{search space}. The search space has the form of a tree, where each branch is a selection of a variable where the variables domain is reduced into a smaller subset that conforms with the constraints. The solver traverses the tree in search for a solution. When all variables are set to conform with the constraints a solution is found. If the solver reaches a node where a variable domain becomes empty, it has to \emph{backtrack} to a previous node from which it can choose a new variable to set, i.e.\ traversing a new branch of that node. To make sure that all constraints holds true, called \emph{consistency}, when a change occurs in a variable during search or by propagation itself, the solver performs \emph{propagation}. When a change occurs to a variable, lets call it $X$, the solver looks at the variables related to this variable through constraints, lets call the $Y$ and $Z$, and may prune the domains of $Y$ and $Z$ in order to uphold the consistency of the constraints. The solver then propagates onward to the variables related to $Y$ and $Z$ and performs the same procedure.

\cite{tsang_1993}
\cite{marriott_1998}
\cite{mz_manual}

\subsection{Reified Constraints}
Reified constraints are constraints that couple a primitive constraint with a boolean variable and provide a relationship between the both. An example could be the constraint $c$ and a boolean variable $B$, the relationship between the both could be $c \Leftrightarrow B$. This says that if $c$ holds, $B = true$ and if $\neg c$ holds, $B = false$. This form of expression can be very useful in expressing complex relations and constraints \cite{marriott_1998}.

Although it is a convenient way of expressing complex relations, it has its disadvantages. Reified constraints can be inefficient since every reified constraint needs to be propagated all the time. Reified constraints can also propagate poorly, for example if a variable occurs multiple times in an expression \cite{jefferson_2010}. Due to this, we have tried to avoid direct reified constraints in the MiniZinc code in hope of reducing the total amount of reified constraints in the resulting code.

\subsection{Branching Heuristics}
Branching heuristics is what decides what value in a domain to branch on. It can be declared by the one programming the model and can play a significant role in the effectiveness of the model. An example of a common branching heuristic is \texttt{indomain\_min} which branches on the smallest value in the domain and if backtracked choses the next smallest value the next time, i.e.\ working its way up from the smallest value. The opposite of \texttt{indomain\_min} is \texttt{indomain\_max}, which starts in the other end of the domain. Another common branching heuristic is \texttt{indomain\_median} which branches on the median value of the domain and if backtracked branches on the values on either side of the median and works its way outwards. There are more branching heuristics available and which branching heuristics are available depend on the solver.

\section{Job-shop scheduling problem}
The job shop problem can be described as $n$ jobs of varying size containing a number of operations to be executed in a certain order that needs to be scheduled on $m$ identical machines. Commonly the goal is to minimize the total time for the schedule, called the \emph{makespan}. The traveling salesman problem is a version of the job shop problem where $m = 1$. \cite{garey_1976} shows that the job shop problem is NP-complete for $m \geq 2$ and $n \geq 3$, hence more complex versions of the job shop problem will be at least this hard.

As described above, the schedule is composed of jobs containing operations. This is the usual way of describing it in the literature, but we will look at it in a slightly different way. Instead of looking at many jobs, we will focus on one job and the operations within that job. In this thesis we will refer to these operations as tasks.
\\\\
An extension of the job shop problem is the flexible job shop problem. In it, tasks are not locked to be scheduled on a particular machine, but can be scheduled for any of the machines \cite{thornblad_2013}. This increases the complexity of the problem.
\\\\
Yet another extension of the job shop problem is the job shop problem with sequence-dependent setups. This means the time for a task is not just the time it takes to execute the task itself, but also the time it takes to set up the machine, depending on the previous task, in order to execute the task at hand. This is also something that increases the complexity compared to the basic job shop problem.
\\\\
Our case will be a combination of the flexible job shop problem and the job shop problem with sequence-dependent setup times since, as will be described later, we can change the tools of the machines. The ability to change tools means that all machines can execute all tasks, and the change of tool takes time which means we get a sequence-dependence.

\section{MiniZinc}
There are many solvers for CP problems, but they all use different languages and as a modeler it might be of interest to test how well your model performs on different solver. To eliminate the need to rewrite models to fit the language of the different solver in order to perform a comparison, MiniZinc was introduced. MiniZinc is a modeling language similar to \emph{Optimized Programming Language} (OPL), but is scaled down and lacks some of OPL's features. MiniZinc's strength lies in that it is coupled with another language called FLatZinc. The difference between MiniZinc and FlatZinc is that MiniZinc is a medium-level language where it is easy for modelers to express themselves and FlatZinc is a low-level language that is easy for interpreters to parse. There is a translator from MiniZinc to FlatZinc provided, the translation is called \emph{flattening}. MiniZinc provides a set of already defined constraints that solvers can use, however, the translator takes in consideration the solver that is going to be used and can apply custom versions of the constraints specified for that particular solver.
\cite{mz_paper}

Although MiniZinc aims at being a standard language in CP, it does not have support for implementing custom search algorithms, as many other languages do. This means we cannot utilize algorithms for random restart, local search, etc.
\cite{mz_paper}

\section{Solvers}
This thesis will test the model using 6 different solvers; \emph{G12}, \emph{JaCoP}, \emph{Gecode}, \emph{OR-tools}, \emph{Opturion CPX} and \emph{Choco3}. There where three requirements considered when we chose the solvers:
\begin{itemize}
\item The solver has to have a FlatZinc parser
\item The item has to be free to acquire, either via open source, free license or free academic license.
\end{itemize}

The model was initially tested during the implementation phase using G12/FD, but after a while G12/FD was unable to produce results and a switch was made to JaCoP. In other words, the model was developed and initially tested using JaCoP.

\subsection{G12/FD}
\emph{G12/FD} is a finite domain solver provided by the G12 team, the creators of MiniZinc. It is implemented in Mercury and is the default solver for the G12 FlatZinc interpreter. \cite{nicta_2964} \cite{mz_result_2014}
\subsection{JaCoP}
JaCoP stands for Java Constraint Programming solver, and is an open source Java library for constraint programming that is available under the GNU Affero GPL license. It has been developed since 2001, mainly by Krzysztof Kuchcinski and Radoslaw Szymanek. The library provides many global constraints in order to make modeling more efficient. It is used by researchers all around the world and has proven its efficiency by winning silver prize in the MiniZinc Challenge.
\cite{jacop_overview}
\cite{jacop_about}
\subsection{Gecode}
Gecode is a free constraint solver under the MIT License implemented in C++. It officially provide a MiniZinc interface, but many external projects provides additional interfaces. One of its strengths is that it can perform parallel searches using multiple cores and it gives the solver great efficiency. This has lead to Gecode winning all the gold medals of the MiniZinc challenge in all 5 consecutive years between 2008 and 2012.
\cite{gecode}
\subsection{or-tools}
or-tools is an open source constraint solver under the Apache License 2.0 implemented in C++. or-tools is developed by Google and is part of their Operational Research. In addition to C++ and MiniZinc, or-tools also has interfaces for Python, Java and C\#.
\cite{or_manual}
\subsection{Opturion CPX}
Opturion CPX is a constraint solver developed by Opturion Pty Ltd, a commercial outcome of the G12 project. The same ones that created G12/FD, MiniZinc and FlatZinc. Opturion CPX is a commercial product and therefore not free. Although, they provide academic licenses which was used for the thesis. Since it originated from G12, the language for implementing models is MiniZinc.

Unlike the other solvers used, Opturion CPX is not a pure \emph{finite domain} solver, but rather a combination of solving techniques from CP and propositional logic (SAT). This makes CPX extremely efficient in solving large models.
It is said that because Opturion CPX only generates propositional variables needed for the search, the search is not necessarily slowed down due to large domains.
Proof of this can be shown by the number of awards claimed in the MiniZinc challange.
\cite{cpx}
\cite{cpx_about}
\cite{cpx_site}
\subsection{Choco3}
Choco3 is a finite domain \cite{choco_paper} constraint solver implemented in Java and it is free under the BSD license. The development of Choco has been going on since the early 2000s and Choco3 is the latest version. Although sharing the name, Choco3 is not the same system as its predecessor Choco2, but a complete rewriting of the previous system \cite{choco}.