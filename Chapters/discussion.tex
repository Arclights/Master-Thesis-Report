\chapter{Discussion}\label{cha:discuss}

\section{Model}
As we mentioned earlier, the design of this model has taken inspiration from Ejenstam's model \cite{ejenstam_2014}. Although, there are some differences in our case studies that affect our models in different ways.

In Ejenstam's model each sub assembly has its own dedicated fixture. Or rather, each sub assembly that does not require any lifting from the fixture. Where the limit goes is not clear. Take for example the case of putting \emph{component1} in a fixture, mounting \emph{component2} on \emph{component1}, pick upp sub assembly \emph{component1-component2}, do some operation, maybe turning it over, putting it back in a fixture, and mount \emph{component3} on sub assembly \emph{component1-component2}. Do we need two fixtures or does one suffice? In our model we can cope with only one fixture even if we have more than one sub assembly, as in our case study. But this also means we need to ensure that no part is put in each fixture before the fixture has been emptied. We do this with constraint \ref{eq:100} where we identify which put and take is associated with each sub assembly on the fixtures. One does not need to do that in Ejenstams model since there is as many fixtures as sub assemblies and therefore automatically only one put and one take task for a single fixture. It makes Ejenstams model simpler in this aspect, since the problem of assigning which sub assembly gets which fixture and when it gets it is solved before the solver tries to solve the problem. In our model it is part of the problem for the solver to solve.

In both Ejenstam's model and our model we try to perform collision avoidance. By that we mean that we avoid collisions in, for example, fixture, it does not mean that we avoid collisions all together. We do not assure that arms does not collide in mid air. Although, it is somewhat accounted for when we say that arms cannot reach certain tasks. Since the robot has two arms positioned in such a way that we could divide the work area in two parts where one arm is responsible for one part and the other arm for the other part. In this way we could avoid them colliding when moving about. Although, this might not be the best strategy when aiming for the smallest makespan. The feature of being able to assign tasks that certain arms cannot reach was developed last in our model and right before implementing that feature we got the makespan of 504 time units. Compared to the result we got after that feature we would get a better makespan if the feature was not used. This is hinting that it could be beneficial to not constrain the tasks the arms cannot reach too much.

Another thing that needs to be considered to avoid the machines from colliding is where they go when a task s completed. Because there can be a wait between when a task ends and the next task starts the machines can be stationary at some point in space. This is nothing our model takes into consideration, instead our model assumes that the machine performing the task does not remain at the point where the task was performed after the task ends, as can be seen in constraints such as \ref{eq:104}. Instead it is something the one who creates the data to be scheduled needs to think about. The one creating the data might create tasks in such a way that they actually represents the task and a move away from the task to a spot where the machine will be able to wait for the next task. This will not eliminate the problem of the other machine colliding with the machine when it waits in the new spot. But it will eliminate the problem of machines colliding at, for example, a fixture when one arm has put a component in the fixture and the other machine wants to mount another component.

Instead of be able to change tool as in our model, Ejenstam's model gives the possibility to have many tools on the same arm at the same time. Our ambition was to be able to solve the data for Ejenstam's case study using our model. And thanks to Johan Wessen at ABB we were allowed access to the data used. Unfortunately, part of the functionality of Ejenstam's model that put a spanner in the works for that ambition. Because if we wanted to solve Ejenstam's data we would need to be able to have hands with many tools. As it is now we can have one tool mounted on a machine at a time with the possibility to switch to another tool. If we have a case were we have \emph{tool1} mounted on the machine but we need \emph{tool2} there is only one thing we can do, and that is to change the tool. If we incorporated the possibility to have many tools on one hand and have the case of having three hands were \emph{hand1} having one \emph{tool1}, \emph{hand2} and two \emph{tool2} and \emph{hand3} having one \emph{tool2} and one \emph{tool3}, having \emph{hand1} mounted on the machine and needing \emph{tool2}. This gives the possibility to either switch to \emph{hand2} or \emph{hand3} and thereby increasing the complexity of the problem because a decision on what hand to mount on the machine at this time will affect the need to change hands in the future in a greater way compared to changing single tools as we do in our model.

By only allowing one tool mounted at a time in our model we can filter the precedences of certain types of tasks. For example in our model, because of only having one tool at a time, we can only pick up one component at a time and we have to put it down or mount it before picking up another tool. It is in contrast to Ejenstam's model where we can pick up several components after one another, the amount depending on the hand mounted on the machine. Because of this, we can in our model filter taking tasks so they cannot be predecessors to one another and thereby reducing the search space. We do this filtering with constraint (\ref{eq:75}) and (\ref{eq:76}).

In a way our model is more complex than Ejenstam's because we have the options to customise mountings and movements, such as mounting in mid air using only the two machines and no fixture. Or use fixtures to store components or sub assemblies. But Ejenstam's search space is much larger, mainly because of two reasons. One, they want to not only find an optimal solution using one fixed setup of trays, but rather how can we place the trays to get the most optimal makespan. Which is like our case study, but with another layer on top that needs to be solved. And two, because they can pick up several components after one another before mounting or putting them down. These two factors makes Ejenstam's search space much larger than ours and thus it does warrant for using customised search strategies as the ones tested in the thesis. Since out case study is relatively small it might not necessarily need some customised searches. But as our results shows, this seem to vary depending on what solver is used. And also, if the assembly would be larger, i.e. more tasks to perform, such as if we would like to scheduled several cycles of the assembly, there might come a need for using customised searches such as the ones in Ejenstam's thesis.

As opposed to our case study, Ejenstam made a time study where he measured the times both for the time of the moves between tasks and for the time of the tasks themselves. The moce times where measured using RobotStudio\cite{robotstudio}, whether or not RobotStudio was used for measuring the task times is unclear. While we on the other hand estimated our times using a video. Ejenstam used the exact measured times from the time study and by looking at the data from the study one can see that the times vary some on the same task depending on what machine performed the task. This might seem a bit odd, but it could be due to an unknown factor that might have to be considered when a task is performed. It could also be a coincident when the measurements were made. If this is the case, measurements that is prone to errors that will make the times between arms on the same task different from one another has the potential to introduce unfair advantages to one of the machines. In an ideal environment a task should take the same amount of time to perform independent of the machine performing it. This is how we look at it, we use the same times for both arms. Although, as said before, we use estimated times, so our times are even more error prone than Ejenstam's. But our times does not have the potential to give unfair advantages to one machine or another. But it is a big source of error when it comes to comparing the time from the solver and the real assembly in the video. Since we first estimated the times and then modified them a bit, we could not simply compare the results from the solvers with the time in the video. But rather, we had to analyse the video again and try to append the estimated times on the tasks in the video and when they occurred and get a total time of the assembly that way. In future works it would be beneficial if a time study was performed for the use case.
\\\\
In our model we represent tasks with that needs multiple machines as as many tasks as there are machines needed and that they need to be executed simultaneously. Another possible way to model it would be to have one task with many machines. But that would mean that tasks could have many predecessors since there is as many tasks previous to this tasks as there are machines performing this task that needs to be linked with this task. And it would also require all the precedence constraints to be remodeled and the \texttt{circuit} constraint would not work for this model anymore since tasks would now be able to appear in multiple places on the circuit. So the approach of using multiple tasks to model a task using multiple machines might need additional information from the input, but in all is easier to model with constraints.

In our input to the model we provide the time in a matrix form with all the possible times needed already calculated. This seems to give the solver a good performance since it does not need to calculate it on the fly over and over again. We do a similar thing with the time matrix file we use to generate the input file to the solvers, see \ref{sec:time_matrix}. Since there are many tasks performed in the same physical location, as in our case study where all fixture related tasks are performed in the same fixture, a change in the location of the fixture would make it a hassle to change all the affected times. We could solve it in the XML file, see \ref{sec:xml}, by associating all the locations used with keys and then each task would be related to one of those keys. Then the translation tool, see \ref{sec:assemblyConv}, could look up the position for the tasks and calculate the time the time for the movement between them.

There is a filter applied to the $moveDuration$ for each task that limits its domains, the (\ref{eq:67}) constraint. This filter might not be necessary since the values for $moveDuration$ is assigned using the $3DTimeMatrix$ and therefore not searched like a regular domain. So limiting the domain to just the values it can be assigned from the matrix probably does not make any difference.

\section{Results}
Nämn något om makespan resultatet

* Att köra själv fz medl fzn-fil tar mycket längre tid än att låta skriptet kompilera och köra direct -> går inte att mäta tiden för endast lösning och utan parsning -> vi kör alla test med parsning

* Anledningen till de stora filerna för G12 Lazy och Opturion CPX kan bero på att de använder sig av SAT. Båda är hybrider, Lazy SAT och FD [T.Feydy,P.J.Stuckey] och Opturion SAT och något odefinierat(Kolla upp!). Kolla mer i literatur

Skillnader mellan tider för mauell assembly och beräknad assembly beror på vissa constraints som gör att tasksen inte kan linea upp precis som i den manuella assemblyn, beroende på hur move till en task kan börja relativt till när relaterade tasks avslutas.

As mentioned, in 1.6 the filters increase the number of constraints quite a bit. But the number of reified constraints are decreasing by quite a lot as well. This seems to indicate that the filters does not introduce more reified constraints, which is good.

Decrease in constraints could be connected to the difference in number of constraint when applying filters in 1.6 vs 2.0.1.

in 2.0.1 temp filter adds no constraints, pred filter adds small amount -> hints of pruning instead of relying on the solver to prune



