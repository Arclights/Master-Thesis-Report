\chapter{Introduction} 

More and more of the production in today's society is getting automated.
Product series have short lifespan and the focus of the production must change quickly. It is expensive to have at hand robots for every possible occasion, thus such production is often outsourced to low-wage countries. Often with worse working conditions than in the west. This puts pressure on the robot manufacturers to develop robots that are versatile like humans and thus eliminating the need to have multiple robots to do multiple tasks which will lower the costs and close the gap of what a human and robots are able to do.

Current robot setups usually have one robot performing one task all the time, as opposed to flexible robots which will be changing between many different tasks and assemblies. One of these flexible robots is ABB's robot YuMi\textsuperscript\textregistered. YuMi\textsuperscript\textregistered is a dual armed robot made to work alongside humans and able to perform some of the most complex tasks, such as mount a nut or thread a needle\cite{_yumi_}. It accomplishes this by using a wide variety of sensors, e.g., force sensor, visual sensors, etc. Usually a robot replaces humans to perform dangerous or heavy tasks, while YuMi\textsuperscript\textregistered is mainly designed for small parts cooperation with humans, i.e. usually human roles in todays manufacturing environment.

In order to support rapid change-over between tasks, certain problems such as scheduling of the assemblies need to be automated. The problem of scheduling is a classic constraint problem. Hence, in this thesis we will be using Constraint Programming (CP) to automate the scheduling process in order to cut down on the scheduling time. Constraint Programming provides a general interface to solve problems without needing to build a complete framework from scratch and makes it easy to formulate the problem through a model.

\newpage

\section{Project goal}
The goal of this thesis is to present a generic CP model suitable for a robot such as YuMi\textsuperscript\textregistered, able to handle the type of jobs YuMi\textsuperscript\textregistered is able to perform. The scope of the thesis will cover assemblies where the robot can change tools, but only being able to pick up one object at a time. Also, the change between two tools will take the same amount of time regardless whether it is from tool 1 to tool 2, or the other way around. The model will cover the use of trays, fixtures and outputs.

The model will be constructed using the MiniZinc language and tested with 6 CP solvers. We will compare the results from the solvers, both to see how well our model can perform and how well the solvers perform relative to one another.

\section{Related work}
\cite{sethi_2006} conclude that the increasing number of machines in a robotic cell causes an explosive growth in combinatorial possibilites. They also provide evidence that a dual-gripper cell is more productive than a single-gripper cell.

\cite{thornblad_2013} concludes that when a cell is part of an assembly flow, the targeting of due dates instead of makespan, the total time for the assembly, is to prefer. The reason is that the focusing on makespan runs the risk of exacerbating an already unreliable flow. However, the assembly we want to construct is not a part of a flow, and thus we do not concern ourselves with maintaining a stable flow through the cell, but only to optimize the assembly in the cell.

\cite{yuan_2013} states that Constraint Programming is only effective on small problems of flexible job shop scheduling. In order to effectively solve problems of larger size they suggest to use methods such as large neighbourhood search (LNS) or iterative flattening search. They also show that LNS together with Hybrid Harmonic Search produces good results.

Unfortunately MiniZinc does not support the implementation of custom searches such as LNS. Therefore we try to solve the problem of ineffectiveness by using filter such as those presented by Vilím in \cite{VilimBartak2002Batch} \cite{Vilim2002Precedence} and \\\cite{VilimBartak2002Sequence}.

\cite{ejenstam_2014} conducted a similar study also on the YuMi\textsuperscript\textregistered robot. In the study Google or-tools was used to write the model and also implemented \emph{Systematic Tree Search}, \emph{Random Restart} and \emph{Local Search} and compared the results of using different combinations of them. The case study and goal of the study is different from this thesis and therefore the resulting models differ, as is discussed in chapter \ref{cha:discuss}. In the study it was conclude that \emph{Systematic Tree Search} combined with \emph{Random Restart} produced schedules with better results than the reference solution.
\\\\
Unfortunately, not many comparisons between MiniZinc compatible solvers where found. There is an annual competition held by NICTA where solvers can compete, this is the most comprehensive documentation of the performance of the solvers we have found. Unfortunately, they only present which solver wins a category and no statistics are presented. Hence, no deeper comparison can be made from the result. In the latest competition held, 2014, or-tools won three out of the four gold medals, Opturion CPX won all four silver medals and Choco won three out of the four bronze medals\cite{mz_result_2014}.

When presenting MiniZinc for the first time, initial tests were also presented comparing, amongst others, G12/FD, Gecode using FlatZinc code and native Gecode. The tests show that MiniZinc was competitive with the native Gecode model and on average the Gecode front-end for FlatZinc was about 200ms faster than the G12/FD\\\cite{mz_paper}.

Another comparison found was \cite{nicta_2964} where they tested 10 solver on 12 problems. Unfortunately, the only solver tested there that we also used in this thesis is the G12/FD solver. So comparison with our results is hard. Although they did not draw any conclusions, G12/FD seem to fare relatively well compared to the other solvers tested.


\section{Report structure}
First we will present the approach we have taken in the thesis and present the relevant background information in chapter \ref{cha:approach}. In chapter \ref{cha:assembly} we will present the case study assembly used in the thesis. Then we will present an in depth view of the model created in chapter \ref{cha:model}.  In chapter \ref{cha:eval} we will present the setup used to evaluate the model and the result of the evaluation. In chapter \ref{cha:discuss} we will discuss the results and we will come to a conclusion in chapter \ref{cha:conc}.
\\\\
Lastly we have two appendices. Appendix \ref{app:ext_mod} contains constraints which is not crucial for solving of the problem, but is still a part of the model. In appendix \ref{app:tool_manuals} we present all the tools used, which are free to use, and where to acquire them.
