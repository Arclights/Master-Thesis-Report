\chapter{Introduction} 

\section{Background}
More and more of the production in today's society is getting automated. Manufacturers want to cut cost and make the production more effective by eliminating the human work and replace it with robots. But there are drawbacks; robots are expensive and robots does not have the versatility of a human. This puts pressure on the robot manufacturers to develop robots that are more versatile. Thus eliminating the need for manufacturers to have multiple robots to do multiple tasks and thereby lowering costs. And also by making the robots more versatile the close the gap of what a human and robots are able to do.

Current robot setups usually have one robot performing one task all the time, as oppose to flexible robots which will be performing many different tasks and assemblies. This poses the demand for the scheduling of such robots to be flexible as well. A scheduling of a robot can be a time consuming task. Since manufacturers want as effective assemblies as possible, it can take from days to weeks to perfect an assembly schedule. This is not feasible if you want to use the robot for many different tasks and assemblies. In this thesis we would like to try and automate this scheduling process in order to cut down on the scheduling time. To accomplish this we will be using Constraint programming, as it provides a general interface to solve problems without needing to build a complete framework from scratch. Also, scheduling is a classical constraint problem, thus constraint programming suits this problem well.

One of those robots are ABB's robot YuMi\textregistered(formerly known as FRIDA). YuMi\textregistered is a dual armed robot made to work along side humans and able to perform the some of the most complex tasks, such as mount a nut or thread a needle.\cite{_yumi_} It accomplishes this by using a wide variety of sensors, e.g. force sensor, visual sensors, etc. Usually robot replaces humans to perform dangerous or heavy tasks, YuMi\textregistered is mainly designed for small parts assembly, i.e. usually humans roles in todays manufacturing environment.





\section{Problem specification}
What does our problem look like

\section{Related work}
[brucker 2009] mentions the solving of 15x15 benchmark(15 jobs with 15 operations) being solved without heuristics (although not necessarily in CP).\cite{brucker_2009}
\\

[Thörnblad] concludes that when a cell is part of an assembly flow, the use of targeting due dates is to prefer. Because makespan can exacerbate an already unreliable flow. The assembly we perform is not a part of a flow, and such we do not concern ourselves with maintaining a stable flow through the cell, but only to optimize the assembly in the cell.\cite{thornblad_2013}
\\

[Garey] shows that job shop problems for size $m \geq 2$ and $n \geq 3$ are NP-complete \cite{garey_1976}
\\

[yuan] says pure CP is only effective on small problems of FJSP. To effectively perform larger sizes, methods such as discrepancy search, large neighborhood search(LNS) or iterative flattening search. It also shows LNS together with Hybrid Harmonic Search produces good results.\cite{yuan_2013}
\\

[ejenstam]\cite{ejenstam_2014}

[andra test med solvers]\\
During construction of MiniZinc, initial tests comparing, amongst others, G12 and Gecode\cite{mz_paper}

Compared solvers included in MiniZinc, including G12/FD, unfortunately no more that we use\cite{nicta_2964}

\section{Report structure}
